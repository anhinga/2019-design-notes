\documentclass{article}

\usepackage{hyperref}
\usepackage{comment}

\baselineskip=15pt minus 1pt
\overfullrule=0pt

% maximize page usage
\oddsidemargin 1pt
\evensidemargin 1pt
\marginparwidth 30pt % these gain 53pt width
\topmargin 1pt       % gains 26pt height
\headheight 1pt      % gains 11pt height
\headsep 1pt         % gains 24pt height
%\footheight 12 pt % cannot be changed as number must fit
\footskip 24pt       % gains 6pt height
\textheight % 528 + 26 + 11 + 24 + 6 + 90 for luck
            685pt
\textwidth % 360 + 53 + 47 for luck
           480pt
% end of page layout changes

\begin{document}

\renewcommand{\abstractname}{\vspace{-\baselineskip}}

\renewcommand\contentsname{\vspace{-\baselineskip}}


\begin{center}

{\bf Dataflow Matrix Machines:  a Collaborative Research Agenda}
                                   



\vspace{0.1in}
Michael A. Bukatin


\vspace{0.085in}
December 22-24, 2019 (in progress)

\end{center}



\begin{abstract}

{\bf Dataflow matrix machines} form a quintessential interdisciplinary field. 
They emerged as a class of neural machines expressive enough to also serve
as a viable programming framework. Their historical roots are in the synthesis of domains
for denotational semantics and vector spaces. There are deep connections between
vector semantics of programming languages and fuzzy and multivalued logic of
partial inconsistency. The dynamical systems based on dataflow
matrix machines exhibit a variety of interesting emerging properties. 

As a programming framework, dataflow matrix machines have affinity with
synchronous versions of dataflow and functional reactive programming.
They generalize digital audio synthesis based on composition of unit generators
(transformers of streams of numbers), thus providing potential to generalize
the style of programming via composition of unit generators to
visual animations, virtual reality, and eventually to general-purpose
programming.

A class of spaces of V-values (flexible tensors based on tree-shaped indices) is
extremely convenient for a variety of purposes. V-values allow to bring conventional data structures
into the neural context, are convenient for hierarchies, are used to allow
variadic activation functions, and have good potential for use in creating and training
flexible neural interfaces between pre-existing software systems.

When considered as neural machines, dataflow matrix machines are remarkable for
their strong self-referential facilities, allowing a neural network of this class to analyze
and modify its current configuration on the fly. They form a natural framework for modular
neural networks. This suggest a strong potential for their use in learning to learn,
and also in neuroevolutionary methods.

Dataflow matrix machines were discovered and studied by a series of small-scale academic
research collaborations. To unlock their full interdisciplinary potential, it would be necessary
to generate a wider interest in this class of programmable neural machines.

\end{abstract}

\begin{center}
\line(1,0){250}
\end{center}

In this note, I provide a bit more details and key references for some of the promising
interdisciplinary research directions I see at the moment. I hope readers from various
fields would find this of interest.

\section{Background: how they work}

The essence of neural model of computations is that linear and non-linear computations are interleaved. Hence, the natural
degree of generality for neuromorphic computations is to work not with streams of numbers, but with arbitrary streams
supporting the notion of linear combination of several streams ({\bf linear streams}).

Dataflow matrix machines (DMMs) form a novel class of neural machines, which work with wide variety
of {\bf linear streams} instead of streams of numbers. The neurons have
arbitrary arity (arity of a neuron can be fixed or variable). Of particular note are
self-referential facilities: ability to change weights, topology, and the size of the active part of the network dynamically, on the fly,
and the reflection capability (the ability of the network to analyze its current configuration).

There are various kinds of linear streams. They include streams of numbers, sparse vectors and sparse tensors (both of
finite and infinite dimension), streams of functions and distributions. We found streams of V-values
({\bf flexible tensors} based on tree-shaped indices) to be of particular use.

A single dataflow matrix machine can process a large variety of different kinds of linear streams, or
it can be based on a single kind of linear streams, sufficiently expressive for a given class of situations.

This allows us to obtain 
neural machines which combine {\bf general-purpose programming powers of stream-oriented
architectures} such as traditional dataflow programming and
more novel functional reactive programming with {\bf good machine learning
properties of conventional neural networks.}

\vspace{0.1in}
\noindent
{\bf Dataflow Matrix Machines resources:}

Reference paper: \href{https://arxiv.org/abs/1712.07447}{\tt https://arxiv.org/abs/1712.07447}

Reference slide deck: \href{https://researcher.watson.ibm.com/researcher/files/us-lmandel/aisys18-bukatin.pdf}{\footnotesize\tt https://researcher.watson.ibm.com/researcher/files/us-lmandel/aisys18-bukatin.pdf}

GitHub Pages: \href{https://anhinga.github.io/}{\tt https://anhinga.github.io}

Open source implementation (Clojure): \href{https://github.com/jsa-aerial/DMM}{\tt https://github.com/jsa-aerial/DMM}

\section{Conventional programming and program synthesis} The dimension of the network and the dimension
of data are decoupled, so compact neural machines for solving conventional programming problems are available.
For example, by considering streams of maps from words to numbers, one can build a dataflow matrix machine
counting words in a given text which uses only a few neurons 
(Section 3 of \href{https://arxiv.org/abs/1606.09470}{\tt https://arxiv.org/abs/1606.09470}).
Similarly, by considering streams of V-values  (flexible tensors based on tree-shaped indices) and embedding
of lists into trees, one can build a similarly compact dataflow matrix machine
accumulating a list of asynchronous incoming events
(e.g. mouse clicks, see Section 6.3 of the DMM reference paper, \href{https://arxiv.org/abs/1712.07447}{\tt https://arxiv.org/abs/1712.07447}). 

The task of synthesis of dataflow matrix machines
should be more tractable than conventional program synthesis. When one works with DMMs, the task of
{\em learning program sketches} is reformulated as {\em neural architecture search},
and converting a program sketch to a full program should be done by
conventional methods of neural net training. 

\vspace{0.1in}
\noindent
Dataflow matrix machines should potentially allow to combine

  \begin{itemize}
      \item aspects of {\em program synthesis} setup\\ (compact, human-readable programs);
      \item aspects of {\em program inference} setup\\ (continuous models defined by matrices).
  \end{itemize}

\section{Self-modification, learning to learn, and neuroevolution} Using neural networks for metalearning
is always non-trivial. In particular, dimension mismatch, namely the number of neuron outputs 
being much smaller than the number of network weights,
means that a neural network
can only modify itself in a highly constrained manner. Dataflow matrix machines address
this problem and have {\bf powerful and flexible self-modification facilities}.

Therefore, a dataflow matrix machine can be equipped with a variety of primitives
which perform self-modifications, and it can fruitfully learn various linear combinations and
compositions involving those primitives.

Self-modification facilities of dataflow matrix machines are not limited to the weight
changes for the existing connections in the network. The available primitives allow to
modify the network topology as well. For example, primitives allowing the network
to control its own fractal-like growth by the means of cloning its own subnetworks
are available.

Therefore, this is a very promising architecture not only for methods of learning to learn
better in a traditional sense, but also for methods of learning to perform
neural architecture search better. 

A dataflow matrix machine can comfortably host
an evolving population of other DMMs inside itself, so it is
an excellent environment for neuroevolution experiments and, in particular,
for the experiments aiming to learn to evolve better (or to evolve to evolve better).

\vspace{0.1in}
\noindent
In our software experiments, we used self-modification facilities to

  \begin{itemize}
     \item produce controlled wave patterns in the network matrix (see Appendix B.2 of our LearnAut 2017 paper, \href{https://arxiv.org/abs/1706.00648}{\tt https://arxiv.org/abs/1706.00648});
     \item create randomly initialized self-referential DMMs which generated interesting emerging behaviors (see Section 1.2 of our 11-2018 technical report, \href{https://www.cs.brandeis.edu/~bukatin/dmm-notes-2018.pdf}{\tt dmm-notes-2018.pdf});
     \item edit a running network on the fly by sending it requests to edit itself (in particular, this enables {\bf livecoding}, but this is also quite open-ended, since it enables a population of networks to tell each other to modify themselves; of course, the receiving network doesn't have to follow an incoming instruction to self-modify blindly, although in the most simple-minded case it would do so; see Section 1.1 of our 11-2018 technical report, \href{https://www.cs.brandeis.edu/~bukatin/dmm-notes-2018.pdf}{\tt dmm-notes-2018.pdf}).
  \end{itemize}

\section{Dynamical systems based on DMMs and emerging properties}

The dynamical systems based on dataflow
matrix machines exhibit a variety of interesting emerging properties. We conducted two series of experimental studies which yielded interesting emerging properties. These experiments were implemented in Processing programming language.

The first series of experiments was conducted in August 2015 and involved continuous cellular automata (see Section 4 of the preprint introducing the class of neural machines which we later started to call DMMs, \href{https://arxiv.org/abs/1601.01050}{\tt https://arxiv.org/abs/1601.01050}). The following videos show some of the emerging patterns we observed:

\begin{itemize}
   \item \href{https://youtu.be/KZHQxdZUlSU}{\tt https://youtu.be/KZHQxdZUlSU}
   \item \href{https://youtu.be/rulK7l4jS-o}{\tt https://youtu.be/rulK7l4jS-o}
   \item \href{https://youtu.be/-pFil1\_GEA4}{\tt https://youtu.be/-pFil1\_GEA4}
\end{itemize}

The second series of experiments was conducted in 2016-2018 and involved ``pure lightweight dataflow machines" (see Section 1.2 of our 11-2018 technical report, 
\href{https://www.cs.brandeis.edu/~bukatin/dmm-notes-2018.pdf}{\tt dmm-notes-2018.pdf}).
We observed, for example, the following emerging patterns:

\begin{itemize}
   \item \href{https://youtu.be/\_mZVVU8x3bs}{\tt https://youtu.be/\_mZVVU8x3bs} - emerging sleep-wake patterns
   \item \href{https://youtu.be/CKVwsQEMNjY}{\tt https://youtu.be/CKVwsQEMNjY} - emerging oscillations
\end{itemize}

This is a fertile setup to discover new emerging patterns of behavior, to study those patterns mathematically,
and, perhaps, to eventually design novel emerging patterns of behavior.

\section{Synchronous vs. async, and artificial vs. biological neural nets}

Like all neural machines, DMMs tend to be synchronous, because it is much easier to combine
several streams with coefficients in the synchronous model of computations. As such, they have affinity with
synchronous dataflow and functional reactive languages and libraries, such as, for example,
{\tt Lucid Synchrone} programming language and {\tt Yampa} (a Haskell library).

By default, DMMs tend to provide extreme sparseness in connectivity patterns (``sparseness in space"),
but not sparseness in time. 

At the same time, biological networks are asynchronous, providing sparseness in time, just like
general dataflow programming tends to be asynchronous and general functional reactive programming,
in particular, tends to provide sparseness in time.

At the same time, biological neural networks tend to be equipped with various synchronization mechanisms.
E.g. on the lower level, mechanisms such as, for example, {\bf leaky integrate-and-fire} enable combining
asynchronous inputs with coefficients. On higher levels, there are various mechanisms synchronizing firing times
of different neurons (see, for example, research by the group led by Nancy Kopell at Boston University starting from
around 2005 for various mathematical models of those mechanisms).

Reconciling synchronous and asynchronous models of computation, and, in particular, borrowing from
what we know about biological neural nets and bringing it into realm of artificial neural machines is
an important direction. Just like all ``sparseness in time", it is also quite relevant to the field of low energy
neural computations, which keeps growing in importance.

\section{Flexibility vs. parallelization and optimization} 

Extreme flexibility of DMMs is provided by the use of V-values (flexible tensors with tree-shaped indices, 
see Section 3 the DMM reference paper, \href{https://arxiv.org/abs/1712.07447}{\tt https://arxiv.org/abs/1712.07447}, for
the theory of V-values),
sparse connections, and the ability of the network to self-reconfigure on the fly.

Obviously, there is a lot of tension between that and the ability to provide an efficient implementation,
and especially with the ability to provide parallelized, GPU-friendly, and batching-friendly implementation.

The task of doing so is not impossible, but can be quite involved (see, for example, TensorFlow {\tt Fold},
a library for working with dynamic computation graphs). See Appendix F of \href{https://arxiv.org/abs/1610.00831}{\tt https://arxiv.org/abs/1610.00831} for further discussion.

We started the initial design work towards reconciling tree-shaped tensor indices and GPUs/TPUs in
\href{https://github.com/anhinga/2019-design-notes}{\tt https://github.com/anhinga/2019-design-notes}
repository. It is also promising to consider more flexible parallel architectures than GPUs, such
as, for example, architectures based on FPGAs, and also commercial flexible alternatives to GPUs,
which are under development at a number of organizations. 

This is an important and potentially very fruitful area of collaboration between specialists in parallel and efficient implementation
of the algorithms and people focusing on flexible neural architectures.

\section{Links to fuzzy and multivalued logic of
partial inconsistency}

The historical roots of dataflow matrix machines are in the synthesis of domains
for denotational semantics and vector spaces. This synthesis told us that it was likely
that programming with linear streams can be made powerful enough to be used for general-purpose programming.

The key element of that synthesis is the ability to handle partial contradictions (that is, to handle
overdefined elements in addition to partially defined elements). For interval arithmetic,  
this corresponds to introduction of pseudosegments $[a, b]$ with the contradictory property that
$b < a$ (the first known discovery of those overdefined interval numbers is by Mieczyslaw Warmus in his ``Calculus of Approximations", 1956). For probability theory, this corresponds to allowing negative values of probabilities,
which originally comes from physics (such as quasiprobabilities by Eugene Wigner in 1932, and then phase-space formulation
of quantum mechanics independently developed by José Moyal and Hilbrand Groenewold in 1940-es).

For the detailed overview of that material linking vector semantics with partial inconsistency, see Section 4 (and, in particular, Sections 4.2-4.4, and Sections 4.7-4.12) of our GCAI 2015 paper, ``Linear Models of Computation and Program Learning", \href{https://easychair.org/publications/paper/Q4lW}{\tt https://easychair.org/publications/paper/Q4lW}. Because in addition to Lawvere duality between fuzzy orders and quasi-metrics, a similar duality exists between multivalued equalities and partial metrics ( \href{https://www.cs.brandeis.edu/~bukatin/distances_and_equalities.html}{\tt https://www.cs.brandeis.edu/$\sim$bukatin/distances\_and\_equalities.html}), here we have both partial metrics and fuzzy multivalued predicates taking values in Warmus' partially inconsistent interval numbers.

There are interesting possible connections to be explored between this material and probabilistic logic networks by Goertzel et al. (2008).


\section*{This is work in progress, this text is to be further modified}

\begin{comment}
\section{Bits and pieces}

{\bf Neuro-symbolic architectures and hierarchical learning.} The availability of 
{\bf flexible tensors} based on {\em tree-shaped indices} makes this architecture
quite promising for neuro-symbolic methods and for hierarchies.

{\bf Visual animation based on composition of unit generators.} DMMs 
generalize digital audio synthesis via composition of unit generators, but work
with more general streams than just streams of numbers. The old-fashioned
analog video synthesizers also work in the style of composition of unit
generators. Modern computer graphics tends to be imperative, oriented towards
specific data flows implemented inside GPUs, and to require the software practitioners to
focus on manual optimizations. The promise of doing animations via functional reactive
programming is to focus again on semantically meaningful
data flows, and to leave the hardware-oriented optimization to the underlying system.
Dataflow matrix machines aspire to make this promise a reality.

We investigated a variety of programming patterns and primitives in
dataflow matrix machines and performed experiments with various uses of self-referential
facilities.
\end{comment}





\end{document}