\documentclass{article}

\usepackage{hyperref}
\usepackage{tikz}
\usepackage{amssymb}
\usepackage{wrapfig}
\usepackage{enumitem}
\usepackage{verbatim}
\usepackage{endnotes}
\usepackage{epigraph}

\definecolor{myblue}{rgb}{0, 0, 0.9}

\newcommand{\msblue}[1]{{\color{myblue} #1}}

\setlength{\epigraphrule}{0pt}


\title{DMMs for VR and worldmaking;\\ a modest proposal on effects and qualia}
\author{Michael Bukatin}

\begin{document}

\maketitle

\begin{abstract}
We look at visual animation and making virtual worlds via composition of unit generators.

We also make a modest proposal for a way to incorporate real-world effects and even qualia into the dataflow matrix machines framework.
\end{abstract}

\epigraph{From digital audio synthesis to visual animation via composition of unit generators;}

\vspace{-0.3in}
\epigraph{From visual animation to VR and making of virtual worlds via composition of unit generators;}

\vspace{-0.3in}
\epigraph{From making virtual worlds to general purpose dataflow programming via composition of unit generators.}


\vspace{-0.1in}
\section*{Introduction}

This is a very informal research note. I hope it will be of use to us in 2020.

\section{Unit generators with linear streams}

Instead of streams of numbers we consider streams of vectors, streams of matrices, and even streams of V-values
(flexible tensors based on tree-shaped indices). 

Currently, a typical framework for visual effects contains a completely intuitive non-programmer interface for end users,
and, on the other hand, engineers write the internals for such a framework and its plugins in some hard-core low-level way (e.g. using some highly optimized C++/CUDA combimation).

Thankfully, there are some intermediate capabilities, e.g. one often encounters {\bf node editors used to program shaders} inside visual frameworks.
The essence of our proposal is to be able to use "intermediate capabilities" for more than just shaders.

In particular, in digital audio synthesis one can use a node editor to live code the sound via {\bf a composition of unit generators}.

We would like to have the same "intermediate capabilities" in visual animation and, generally, in generation of VR and virtual worlds. 
We believe that moving from streams of numbers to {\bf linear streams} (arbitrary streams admitting the notion of linear combination
of several streams) should give us enough expressive power to do just that.

Namely, we expect to be able to create animations, 3D-animations, and more rich multimodal interactive worlds by
composing unit generators which transform linear streams.

\section{Real time vs async}

Stream-based architectures can be async (and then they often tend to be lazy), or they can be ``real time"
(which typically means trying to keep up with some kind of ``global clock" and often defining computations in terms
of ``ticks" of that ``global clock").

Since we come from audio synthesis, visual animations, and such, we tend to focus on ``real time", synchronous
flavor of stream-based programming (which also corresponds well to the approach of artificial neural networks
\footnote{For artificial neural networks, and also for dataflow matrix machines, it is important to be able to
take linear combinations of incoming streams; this is easy to do, if next elements of those streams arrive in sync, but
requires rather intricate treatment otherwise (e.g. leaky-integrate-and-fire of spiking models).}).

In this sense, among "high brow" functional programming frameworks of this kind, one of our favorite examples is Yampa \footnote{See \href{https://wiki.haskell.org/Yampa}{\tt https://wiki.haskell.org/Yampa} and links in that page.}.
It is based on the notion of stream (which it calls {\tt signal}) as a function from {\tt time}. It is sometimes criticized for not being lazy enough
(too many things are recomputed for the mostly static parts of the interface waiting on rare events), but that's a separate
optimization issue\footnote{Optimization issues here are very important; the key one is how to compile our nice user-friendly composition of unit generators into effective GPU-using code; hopefully, some of the preliminary work done in
\href{https://github.com/anhinga/2019-design-notes}{\tt https://github.com/anhinga/2019-design-notes} will be applicable.}.  

Together with explorations by members of DMM community in recent years, work done by members of Yampa community
provides support for a conjecture that this architecture is expressive enough for, basically, any kind of
real-time programming (including async events and elaborate data structures).

\section{A modest proposal on effects and qualia}

This section is motivated, in part, by my recent reading of a variety of texts around ``hard problem of
consciousness". My common observation (which is so far applicable to a wide variety of texts I've seen) is that
while they vigorously disagree with each other on the nature of first-person experience, they make
absolutely zero progress towards understanding and describing the nature of qualia. One can look for
example at drastically different approaches by, say, Donald Hoffman and Chrstof Koch, and there is
exactly one thing they have in common: zero progress towards description of qualia, zero connection between
their elaborate math and something which might start to incorporate models reflecting the particular
richness of particular qualia.

So, I want to try something else. I want to abstract away from all those different approaches to the
fundamental nature of the first-person experience, and I want to take some steps towards incorporating
qualia into our models, so we can at least start working with them. 

But I want to start with something simpler, with ``effects". Sometimes, ``purely computational models
of consciousness" seem to claim that qualia would emerge ``on their own" in the context of appropriate
computations. It is difficult to imagine how this can be, when it is not true even for something as
simple as ``effects". Indeed, we can have very elaborate computer graphics computations, but unless
there is a connection between the machine performing those computations and something that, for example,
lights up small color lights on a computer monitor, no visible ``effects" would emerge. Moreover, it is
easy to connect the wires incorrectly, so that ``red" in our computational model would activate
small green lights, and vice versa (speaking of the ``inverted spectrum" problem). So, if one wants
``effects", e.g. visual effects in the physical world, one needs to incorporate elements actually
lighting some physical color lights in the physical world (or spraying some paint
in the physical world) into one's computational schemas.

\subsection{Qualia}

I think we should apply the same approach to qualia. We should start including elements representing
qualia in our models. We would need to set aside the question of how those elements are made.
But we would need to talk about some properties of qualia, otherwise all this would not even start
to make sense.

{\bf {\em Conjecture.} Given several qualia, in the same modality, or in different modalities, it is usually
possible to ``combine them with coefficients".}

In the same modality, effects like color mixing might be involved, and with different modalities, usually
only a limited amount of attention is available, and only attended components are reflected as 
perceivable qualia. Just like with effects, not all coefficients need to lead to representable results
(e.g. too high or negative color values might not make sense, depending on your model,
or some normalization procedure might need to be applied to make sure that the results
make sense). But, just like with effects, we would expect that at least combinations with
positive coefficients which sum to 1 do make sense.

So, we assume that the {\bf qualia, in some sense, are like linear streams of DMMs}. We don't necessarily
impose a particular structure on those spaces. E.g. what is the dimension of the space of subjective
colors? This depends quite a bit on whether a particular person wants to state that brown is
a kind of dark orange (then one is probably heading towards three-dimensional space of
subjective colors), or if, contrary to that, one wants to state that brown is a very particular sensation,
quite independent from orange (then one is probably heading towards a high-dimensional or,
perhaps, even infinitely-dimensional space of subjective colors).

So, with effects, one can merge DMMs and physical reality, and with qualia as linear streams
one can merge DMMs and the world of perception (at least for the purpose of thought
experiments). Of course, given that an artistically perceiving agent (e.g. human or animal) is likely to end up
being a part of a DMM contour, this model is starting to describe something real.

We are not making judgement on whether it is possible to have qualia-making contours
inside a digital machine, or what should be mechanisms which would make such contours possible.
We just treat qualia-making contours as black boxes, abstracting from whether these black boxes
are biological, whether they might have purely digital implementation, etc. In case of doubts,
it always helps to ask similar questions about effects (e.g. can a computer just light up the monitor
without physical connection, by ``telepathy"? Perhaps, although most likely this would involve some
wireless protocol and some electromagnetic waves to transmit, so we would call it ``pseudo-telepathy").

We just assume: there is something which can control physical lights, there is also something
which produces qualia feelings in a person. We don't know what that something that
produces qualia feelings in a person is, but we assume that we can (perhaps in a non-unique way)
combine several qualia with coefficients (if there is no ``canonical way" to do so, we just pick some
way to combine). It turns out that this modest proposal is sufficient to allow us to talk about qualia
in the context of DMM models, and to consider investigating this subject both in theory
and in experiments\footnote{Some applied situations
including the relevant safety concerns were studied by us in May-June 2012, and the notes made
during that study are, at the moment, available here: \href{https://dmm.dreamwidth.org/19302.html}{\tt https://dmm.dreamwidth.org/19302.html}.}.

\section*{Acknowledgments} 

There are so many crucial influences here during the period of last 20+ years, that I can
barely start acknowledging them all. They come from communities related to audio-visual arts
and DJ/VJ effects, communities formed around studies of the ``hard problem of consciousness", 
communities around functional programming, and around artificial and biological neural networks,
and, of course, from people who were involved in dataflow matrix machines at various stages.








%\theendnotes

%\tableofcontents


\end{document}